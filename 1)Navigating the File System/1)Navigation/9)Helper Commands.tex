Helper Commands
    Now that we’ve covered the basics of navigating your filesystem from the command line, let’s look at some helpful commands that will make using it easier!

    <<clear>> is used to <<clear your terminal>>, which is useful when it’s full of previous commands and outputs. It doesn’t change or undo your previous commands, it <<just clears them from the view>>. You can scroll upwards to see them at any time.

    <<tab>> can be used to <<autocomplete your command>>. When you are typing the name of an existing file or directory, you can use tab to finish the rest of the name.

    The <<up and down arrows (↑ and ↓)>> can be used to <<cycle through your previous commands>>. ↑ will take you up through your most recent commands, and ↓ will take you back through to the most recent one.

Instructions
    Experiment with the helper commands! Some things you could try are:
        // When your working directory is home/ccuser/workspace/blog, type cd 2 and then use <<tab>> - it should autocomplete upto cd 201. This is because both possible directories (2014 and 2015) start with 201.

        // Use <<ls>> to see what contents are in your working directory. Then use cd with the first letter of one of the files or directories and use tab to autocomplete.
        
        // Use the <<up and down arrows>> to cycle through your previous commands.

        // Use <<clear>> to clear your previous commands and output from the terminal.