Review
    Congratulations! You’ve learned five commands commonly used to navigate the filesystem from the command line. What can we generalize so far?
        // The command line is a text interface for the computer’s operating system. To access the command line, we use the terminal.
    
        // A filesystem organizes a computer’s files and directories into a tree structure. It starts with the root directory. Each parent directory can contain more child directories and files.
    
        // From the command line, you can navigate through files and folders on your computer:
            <<pwd>> outputs the name of the current working directory.
            <<ls>> lists all files and directories in the working directory.
            <<cd>> switches you into the directory you specify.
            <<mkdir>> creates a new directory in the working directory.
            <<touch>> creates a new file inside the working directory.
    
        // You can use helper commands to make navigation easier:
            <<clear>> clears the terminal
            <<tab>> autocompletes the name of a file or directory
            <<↑ and ↓>> allow you to cycle through previous commands