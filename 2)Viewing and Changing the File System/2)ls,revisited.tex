ls, revisited
    The <<ls>> command lists all files and directories in the working directory. We can use ls as is, or attach an option. Options modify the behavior of commands.

    For example:
        ls -a
    
    The command above displays the contents of the working directory in more detail. This command <<displays all the files and directories, including those starting with a dot (.)>>. Files starting with a dot are normally hidden, and don’t appear when using ls alone.

    The -a is called an <<option>>. Options modify the behavior of commands.

    In addition to -a, the ls command has several more options. Here are three common ones:
        -a : <<lists all contents, including hidden files and directories>>
        
        -l : <<lists all contents of a directory in long format, as well as the file permissions>>
        -t : <<orders files and directories by the time they were last modified>>.
        
    Let’s practice using these options below.

Instructions
    1.
    In the terminal, type:
        ls -a 

    Do you see the differences between the outputs of both ls and ls -a?

    2.
    Then type
        ls -t

    Do you see the differences between the outputs of both ls -a and ls -t?

    3.
    In the terminal, type
        ls -l 
    
    Click Next to find out what these columns mean.