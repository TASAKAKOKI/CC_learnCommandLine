PATH Environment Variable
    PATH is an environment variable that stores a list of directories separated by a colon.

    What happens when you type this command?
        $ echo $PATH

        /home/ccuser/.gem/ruby/2.0.0/bin:/usr/local/sbin:/usr/local/bin:/usr/bin:/usr/sbin:/sbin:/bin
    
    Looking carefully, echo $PATH lists the following directories, separated by :
        /home/ccuser/.gem/ruby/2.0.0/bin
        /usr/local/sbin
        /usr/local/bin
        /usr/bin
        /usr/sbin
        /sbin
        /bin

    Each directory contains scripts for the command line to execute. The PATH variable simply lists which directories contain scripts.

    For example, many commands we’ve learned are scripts stored in the /bin directory.
    
        /bin/pwd
    This is the script that is executed when you type the pwd command.
       
        /bin/ls
    This is the script that is executed when you type the ls command.

    In advanced cases, you can customize the PATH variable when adding scripts of your own.

Instructions
    1.
    In the command line, type:
        echo $PATH 
    
    This prints the PATH variable.

    2.
    Let’s confirm that most of the terminal commands we’ve learned so far are in /bin/.

    Type the following into the terminal:
        /bin/pwd 
    
    This should have the same output as the pwd command.

    3.
    Type:
        /bin/ls 
    
    This should have the same output as the ls command. 