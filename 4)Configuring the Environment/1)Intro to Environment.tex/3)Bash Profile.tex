Bash Profile
    A bash profile is a file used to store environment settings for your terminal, and it’s accessible by the name ~/.bash_profile.

    When a session starts, it loads the contents of the bash profile before executing commands.
        The ~ represents the user’s home directory.
        The . indicates a hidden file.
        The name ~/.bash_profile is important, since this is how the command line recognizes the bash profile.

    To open and edit the bash profile, you can use the command:
        nano ~/.bash_profile
    
    When you edit the bash profile, you can add commands to execute every time a new terminal session is started.

    For example, if you have an echo statement in the bash profile, that will echo when a terminal session begins.

    To activate the changes made in ~/.bash_profile for the current session, use this following command:

    source ~/.bash_profile
    This makes the changes in the bash profile available right away without closing the terminal and needing to start a new session.

Instructions
    1.
    Let’s edit the environment settings!

    In the terminal, type
        nano ~/.bash_profile 
    
    This opens up the existing, currently blank bash profile file in nano.

    2.
    In ~/.bash_profile, at the top of the file, type:
        echo "Welcome, Jane Doe" 

    You can use your name in place of “Jane Doe.”

    Type Ctrl + O to save the file.

    Press Enter to write the filename.

    Type Ctrl + X to exit nano.

    Then, once you’ve exited nano and are back in the terminal, press the Enter (or return for Mac) key onto a new line.

    3.
    Finally, to see this greeting immediately, use:
        source ~/.bash_profile 