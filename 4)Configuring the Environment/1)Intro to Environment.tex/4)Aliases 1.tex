Aliases I
    As we mentioned in the last exercise, you can add settings and commands that execute every time a new terminal session is started. One type of setting you can create is called an alias:   
        alias pd="pwd"
    
    The alias command allows you to create keyboard shortcuts, or aliases, for commonly used commands.

    Here, alias pd="pwd" creates the alias pd for the pwd command, which is then saved in the bash profile.

    The pd alias will be available each time we open a new terminal session, and the output of pd will be the same as the pwd command.

    source ~/.bash_profile will make the alias pd available in the current session.

Instructions
    1.
    Let’s continue configuring the environment by adding command aliases.
    Open ~/.bash_profile in nano.

    2.
    In ~/.bash_profile, on a new line, type:
        alias pd="pwd"

    Save the changes and exit out of nano.

    Once you’ve exited nano and are back in the terminal, press the Enter key onto a new line.

    3.
    To make the alias in the current terminal sessions, type
        source ~/.bash_profile

    4.
    Let’s try out the alias. Type:
        pd 
        
    You should see the same output as you would by typing the pwd command.