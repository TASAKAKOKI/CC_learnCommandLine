uniq
        $ uniq deserts.txt 
    <<uniq>> stands for “unique.” It filters out adjacent, duplicate lines in a file. Here uniq deserts.txt filters out duplicates of "Sahara Desert", because its duplicate directly follows the previous instance. The “Kalahari Desert” duplicates are not adjacent, and thus remain.
        $ sort deserts.txt | uniq
    
    A more effective way to use uniq is to call sort to alphabetize a file, and “pipe” the standard output to uniq. By piping sort deserts.txt to uniq, all duplicate lines are alphabetized (and thereby made adjacent) and filtered out.
        sort deserts.txt | uniq > uniq-deserts.txt 

    Here we simply send filtered contents to uniq-deserts.txt, which you can view with the cat command.

Instructions
    1.
    Use cat to output the contents of deserts.txt. Note the duplicate entries.

    2.
    Use uniq to remove adjacent duplicates in deserts.txt.

    3.
        sort deserts.txt | uniq > uniq-deserts.txt 
    Use the command above to use piping to create a new file named uniq-deserts.txt that contains a sorted version of deserts.txt with no duplicates.