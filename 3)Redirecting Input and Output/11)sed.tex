sed
        sed 's/snow/rain/' forests.txt 
    sed stands for “stream editor.” It accepts standard input and modifies it based on an expression, before displaying it as output data. It is similar to “find and replace.”

    Let’s look at the expression 's/snow/rain/':
        s: stands for “substitution.” It is always used when using sed for substitution.
        
        snow: the search string, or the text to find.
    
        rain: the replacement string, or the text to add in place.
    
    In this case, sed searches forests.txt for the word “snow” and replaces it with “rain.” Importantly, the above command will only replace the first instance of “snow” on a line.

        sed 's/snow/rain/g' forests.txt 
    The above command uses the <<g>> expression, meaning “global.” Here sed searches forests.txt for the word “snow” and replaces it with “rain” globally. This means all instances of “snow” on a line will be turned to “rain.”

Instructions
    1.
    Use cat to display the contents of forests.txt

    2.
    Use sed on forests.txt to replace just the first appearance of “snow” on each line with “rain”.

    3.
    Use sed again to replace “snow” with “rain” but this time replace every occurrence of “snow”.

    4.
    Use cat again to see the changes you’ve made to forests.txt.    